\selectlanguage{greek}
\def\<#1>{\textit{#1}}

\begin{abstract}


Στις μέρες μας, οι πολυπύρηνοι επεξεργαστές χρησιμοποιούνται ευρέως και έχουν εισαχθεί σε πολλά προγραμματιστικά περιβάλλοντα. Ο παράλληλος προγραμματισμός  δεν αφορά πλέον μόνο επιστημονικές εφαρμογές για υπερυπολογιστικά συστήματα, αλλά καλύπτει ένα μεγάλο φάσμα εφαρμογών, που περιλαμβάνει και εφαρμογές καθημερινής χρήσης σε \textlatin{desktops} ή ενσωματωμένα συστήματα.

Ένα σημαντικό και καθοριστικό κομμάτι για την επίδοση  κάθε εφαρμογής είναι οι δομές δεδομένων που χρησιμοποιεί. Η μετάβαση από αρχιτεκτονικές ενός πυρήνα σε πολυπύρηνες αρχιτεκτονικές, σηματοδοτεί την ανάγκη εκσυγχρονισμού και παραλληλοποίησης των  βασικών δομών δεδομένων, ώστε να ακολουθούν τις τάσεις του μέλλοντος και να προσφέρουν υψηλή κλιμακωσιμότητα.

Η διπλωματική αυτή αφορά τις δομές δεδομένων, με ιδιαίτερη έμφαση στις ουρές και τους πίνακες κατακερματισμού, και μελετά διάφορους τρόπους παραλληλοποίησης τους με βάση τα προβλήματα που καλούνται να επιλύσουν, τα ιδιαίτερα χαρακτηριστικά τους και την συμπεριφορά τους με βάση το υλικό.

\bigskip

\textbf{Λέξεις κλειδιά}: παράλληλες δομές δεδομένων, ταυτόχρονη πρόσβαση, αμοιβαίος αποκλεισμός, ατομικές εντολές, \textlatin{transactional memory}, \textlatin{FIFO} ουρές, πίνακες κατακερματισμού, κλιμακωσιμότητα, επίδοση.

\end{abstract}
