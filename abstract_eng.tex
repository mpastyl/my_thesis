\selectlanguage{english}

\begin{abstract}

Nowadays, multicore processors have become mainstream and are being used by many programming environments. Parallel programming is no longer about scientific applications run in supercomputers, but covers a wider range of environments, including applications on desktops and embedded systems.

An important and crucial factor of every application is the set of data structures it is build on. The transition form single core to multicore systems marks the need to refactor and parallelize basic data structures in order to support higher scalability.

In this thesis we study concurrent data structures, particularly focusing on FIFO queues and hash tables, with respect to the way the are synchronized, the problems they are dealing with, their special characteristics and their effects on hardware.

\bigskip

\textbf{Keywords}: concurrent Data Structures, concurrent access, mutual exclusion, atomic primitives, transactional memory, FIFO queues, hash tables, scalability, performance.

\end{abstract}
